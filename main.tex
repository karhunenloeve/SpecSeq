\documentclass[12pt, hidelinks]{article}
\usepackage[ngerman]{babel}
\usepackage[a4paper, total={6in, 10in}]{geometry}
\usepackage[dvipsnames]{xcolor}
\usepackage{amsmath}
\usepackage{amsfonts}
\usepackage{enumitem}
\usepackage{amsthm}
\usepackage{mathtools}
\usepackage{xcolor}
\usepackage{quiver}
\usepackage{adjustbox}
\usepackage{tikz}
\usetikzlibrary{babel}
\usepackage{tikz-cd}
\usepackage[unicode]{hyperref}
\usepackage[all]{hypcap}

\usetikzlibrary{angles,calc, decorations.pathreplacing}

\definecolor{carminered}{rgb}{1.0, 0.0, 0.22}
\definecolor{capri}{rgb}{0.0, 0.75, 1.0}
\definecolor{brightlavender}{rgb}{0.75, 0.58, 0.89}

\newtheorem{conj}{Vermutung}
\numberwithin{conj}{section}
\newtheorem{definition}[conj]{Definition}
\newtheorem{remark}[conj]{Anmerkung}
\newtheorem{example}[conj]{Beispiel}
\newtheorem{theorem}[conj]{Satz}
\newtheorem{lemma}[conj]{Lemma}
\newtheorem{proposition}[conj]{Proposition}
\newtheorem{corollary}[conj]{Korollar}
\newtheorem{assume}[conj]{Annahme}

\newcommand{\Z}{\mathbb{Z}}
\newcommand{\ima}{\operatorname{im}}

\usepackage[style=authoryear,backend=bibtex]{biblatex}
\addbibresource{main.bib}

\title{\textbf{Spektrale Sequenzen\\ Leray–Serre spektrale Sequenz}}
\author{
\textbf{Luciano Melodia} \\
Seminar zur Spektraltheorie \\
Lehrstuhl für Mathematische Physik \\
Friedrich-Alexander Universität Erlangen-Nürnberg \\
\texttt{luciano.melodia@fau.de}}
\date{\today}

\begin{document}
\hypersetup{bookmarksnumbered=true,}
\maketitle

\begin{Large}
\tableofcontents
\end{Large}

\section{Homologische Spektrale Sequenzen}
Spektrale Sequenzen können im Allgemeinen über abelschen Kategorien definiert werden, oder weniger allgemein über $R$-Moduln. Wir betrachten in dieser Arbeit die spektralen Sequenzen über $\mathbb{Z}$-Moduln, also die Variante für abelsche Gruppen.

\begin{definition}[Homologische Sequenz]
\label{homologischeSequenz}
Sei $0 \leq r_0 \in \mathbb{R}$. Eine homologische Sequenz $\{A^r_{p,q}, d^r_{p,q}\}_{r \geq r_0}$ besteht aus
\begin{enumerate}[nolistsep]
    \item einer Familie abelscher Gruppen $\{A^r_{p,q}\}_{p,q \in \Z, r \geq r_0}$ und
    \item einer Familie Gruppenhomomorphismen $\{d^r_{p,q}: A^r_{p,q} \to A^r_{p-r,q+r-1}\}$ welche Differentiale genannt werden, so dass für alle $p,q,r$ gilt
    \begin{enumerate}[nolistsep]
        \item $d^r_{p-r,q+r-1} \circ d^r_{p,q} = 0$ und
        \item $E^{r+1}_{p,q}$ ist die Homologie von $A^r$ und $A^r_{p,q} = \ker d^r_{p,q} / \ima d_{p+r,q-r+1}$.
    \end{enumerate}
\end{enumerate}
\end{definition}

Diese Definition einer homologischen Sequenz zeigt deutlich auf, für welche Untergruppen die Quotienten berechnet werden - in einer Art Kettenkomplex mit zwei Indizes. Eine übliche Vereinfachung der Notation wird sich jedoch bei den weiteren Untersuchungen als hilfreich erweisen, indem wir ein $r$ festlegen und alle $A^r_{p,q}$ als ein Objekt $A^r$ betrachten. Diese neu entstandene Gruppe nennt man dann eine bigraduierte abelsche Gruppe. Die Differentiale $d^r_{p,q}$ heißen auch kollektiv $d^r$, oder bigraduierte Differentiale.

\begin{definition}[Bigraduierte abelsche Gruppen]
    Eine bigraduierte abelsche Gruppe $A$ ist eine Familie von abelschen Gruppen $A \coloneq \bigoplus_{p,q \in \Z} A_{p,q}$, welche als direkte Summe indexiert über zwei Indizes aus $\Z$ aufgefasst werden kann. Sind zwei bigraduierte abelsche Gruppen $A, B \coloneq \bigoplus_{p,q \in \Z} B_{p,q}$ gegeben, so ist die bigraduierte Abbildung vom Grad $(c,d)$ eine Sammlung von Gruppenhomomorphismen $f \coloneq \bigoplus_{p,q \in \Z} f^{(c,d)}_{p,q}$.
    \[\begin{tikzcd}
        A & {A_{0,0}} & {A_{1,0}} & {A_{0,1}} & {A_{-1,0}} & \cdots \\
        B & {B_{0,0}} & {B_{1,0}} & {B_{0,1}} & {B_{-1,0}} & \cdots
        \arrow["f", color={rgb,255:red,214;green,92;blue,92}, squiggly, from=1-1, to=2-1]
        \arrow["{f_{0,0}}", from=1-2, to=2-2]
        \arrow["{f_{1,0}}", from=1-3, to=2-3]
        \arrow["{f_{0,1}}", from=1-4, to=2-4]
        \arrow["{f_{-1,0}}", from=1-5, to=2-5]
    \end{tikzcd}\]
\end{definition}

Man beachte, dass wir der direkten Summe keine Bedeutung zuschreiben. Die Reihenfolge der Summation ist also willkürlich. Ebenso könnten wir die einzelnen abelschen Gruppen als Elemente einer Menge auffassen. Seien $A = \bigoplus_{p,q \in \Z} A_{p,q}$ und $B = \bigoplus_{p,q \in \Z} B_{p,q}$ also bigraduierte abelsche Gruppen. 

\begin{definition}[Bigraduierte Abbildung]
    Wir nennen $f^{(c,d)}_{p,q}: A_{p,q} \to B_{p+c,q+d}$ einen Gruppenhomomorphismus vom Bigrad $(c,d)$. Dann ist eine bigraduierte Abbildung vom Bigrad $(c,d)$ ein  Gruppenhomomorphismen $f: A \to B, \bigoplus_{p,q} a_{p,q} \mapsto \bigoplus_{p,q \in \Z} f^{(c,d)}_{p,q}(a) = \bigoplus_{p,q \in \Z} b_{p+c,q+d}$.
\end{definition}

$f$ kann als Sammlung von Gruppenhomomorphismen vom Bigrad $(c,d)$ verstanden werden, die elementweise auf der direkten Summe der abelschen Gruppen operieren. Wir können also sinngemäß schreiben $f = \bigoplus_{p,q \in \Z} f^{(c,d)}_{p,q}$.

\begin{example}
    Sei $R$ ein Ring, dann trägt $R[x,x^{-1},y,y^{-1}]$ die Struktur einer bigraduierten abelschen Gruppe $A_{p,q} = Rx^py^q$ für alle $p,q \in \Z$.
\end{example}

\begin{definition}[Bigraduierte Untergruppen und Quotienten]
    Seien $A,B$ bigraduierte abelsche Gruppen. Dann definieren wir
    \begin{enumerate}[nolistsep]
        \item $A \subset B \colon\Leftrightarrow A_{p,q} \subset B_{p,q}$ für alle $p,q \in \Z$ und
        \item falls $A \subset B$, dann ist $B/A \coloneq \bigoplus_{p,q \in \Z} B_{p,q}/A_{p,q}$.
    \end{enumerate}
\end{definition}

\begin{definition}
    Falls $f: A \to B$ eine bigraduierte Abbildung des Grads $(c,d)$ zwischen bigraduierten abelschen Gruppen ist, dann definieren wir
    \begin{itemize}[nolistsep]
        \item $\ker f \coloneq \bigoplus_{p,q \in \Z} \ker f_{p,q}$ und
        \item $\ima f \coloneq \bigoplus_{p,q \in \Z} \ima f_{p-c,q-d}$.
    \end{itemize}
\end{definition}

Nun können wir die zuvor definierte homologische Sequenz (Def. \ref{homologischeSequenz}) zu einer spektralen homologischen Sequenz erheben.

\begin{definition}
    \label{homologischeSpektraleSequenz}
    Sei $r_0 \geq 0$. Eine homologische spektrale Sequenz $\{E^r,d^r\}$ besteht aus
    \begin{enumerate}[nolistsep]
        \item einer bigraduierten abelschen Gruppe $E^r$ und
        \item einer bigraduierten Abbildung $d^r: E^r \to E^r$ vom Grad $(r,r-1)$, welche Differential genannt wird,
    \end{enumerate}
    so dass für jedes $r$ gilt:
    \begin{itemize}[nolistsep]
        \item $d^r \circ d^r = 0$ und
        \item $E^{r+1}$ ist die Homologie von $E^r$ in Bezug auf die Differentiale $d^r$, d.h.
        \begin{align}
            E^r = \frac{\ker d_r}{\ima d_r} = \frac{\bigoplus_{p,q \in \Z} d^r_{p,q}}{\bigoplus_{p,q \in \Z} \ima d^r_{p,q}} = \bigoplus_{p,q \in \Z} \frac{d^r_{p,q}}{\ima d^r_{p,q}}.
        \end{align}
    \end{itemize}
\end{definition}

Für jedes $r$ wird $E^r$ das $r$te Blatt oder die $r$te Seite von $\{E^r,d^r\}_{r \geq r_0}$ genannt.

\begin{remark}
    Es gibt dieselbe Variante als kohomologische spektrale Sequenz, welche auf analoge Weise wie Def. \ref{homologischeSpektraleSequenz} definiert wird, mit der Ausnahme, dass wir nun $E_r \coloneq \bigoplus_{p,q \in \Z} E^{p,q}_r$ für die Kohomologien schreiben und die Differentiale durch $d_r: E_r \to E_r$ geschrieben werden und vom Grad $(r,-r+1)$ sind. Die homologische spektrale Sequenz wird benutzt, um Homologiegruppen zu berechnen, während die kohomologische Spektrale Sequenz benutzt wird um Kohomologien zu berechnen.
\end{remark}

\begin{definition}
    Sei $\{E^r, d^r\}_{r \geq r_0}$ eine homologische spektrale Sequenz, so dass es für alle $p,q \in \Z$ eine natürliche Zahl $r(p,q) \in \mathbb{N}$ gibt, abhängig von $p,q$, so dass für alle $r \geq r(p,q)$, $E^r_{p,q} \cong E_{p,q}^{r(p,q)}$. Dann sagen wir, dass die bigraduierte abelsche Gruppe
    \begin{align}
        E^\infty \coloneq \bigoplus_{p,q \in \Z} E^{r(p,q)}_{p,q}
    \end{align}
    der Limes für $\{E^r, d^r\}_{r \geq r_0}$ ist. Äquivalent sagen wir auch, dass die spektrale Sequenz an $E^\infty$ angrenzt.
\end{definition}

\begin{remark}
Eine visuelle Darstellung der ersten drei Seiten der Serre spektralen Sequenz mit den Differenzialen ($\rightarrow$) und den Werte- und Zielbereichen dargestellt durch $\bullet$. Die $\bullet$ repräsentieren dabei Gruppen, insbesondere Homologiegruppen, oder Homologievektorräume über einem Körper $\mathbb{F}$. Der rote Punkt $\color{red}\bullet$ entspricht $E^1_{0,2}$ und der blaue Punkt $\color{blue}\bullet$ entspricht $E^2_{1,-2}$.
\[
\begin{tikzcd}[sep=tiny]
    &&& {E^1} &&&&&& {E^2} &&&&&& {E^3} \\
    & \bullet & \bullet & {\color{red}\bullet} & \bullet & \bullet && \bullet & \bullet & \bullet & \bullet & \bullet && \bullet & \bullet & \bullet & \bullet & \bullet \\
    {} & \bullet & \bullet & \bullet & \bullet & \bullet && \bullet & \bullet & \bullet & \bullet & \bullet && \bullet & \bullet & \bullet & \bullet & \bullet \\
    & \bullet & \bullet & \bullet & \bullet & \bullet && \bullet & \bullet & \bullet & \bullet & \bullet && \bullet & \bullet & \bullet & \bullet & \bullet \\
    & \bullet & \bullet & \bullet & \bullet & \bullet && \bullet & \bullet & \bullet & \bullet & \bullet && \bullet & \bullet & \bullet & \bullet & \bullet \\
    & \bullet & \bullet & \bullet & \bullet & \bullet && \bullet & \bullet & \bullet & {\color{blue}\bullet} & \bullet && \bullet & \bullet & \bullet & \bullet & \bullet
    \arrow[from=2-3, to=2-2]
    \arrow[from=2-4, to=2-3]
    \arrow[from=2-5, to=2-4]
    \arrow[from=2-6, to=2-5]
    \arrow[from=3-3, to=3-2]
    \arrow[from=3-4, to=3-3]
    \arrow[from=3-5, to=3-4]
    \arrow[from=3-6, to=3-5]
    \arrow[from=3-10, to=2-8]
    \arrow[from=3-11, to=2-9]
    \arrow[from=3-12, to=2-10]
    \arrow[from=4-3, to=4-2]
    \arrow[from=4-4, to=4-3]
    \arrow[from=4-5, to=4-4]
    \arrow[from=4-6, to=4-5]
    \arrow[shorten <=5pt, Rightarrow, from=4-6, to=4-8]
    \arrow[from=4-10, to=3-8]
    \arrow[from=4-11, to=3-9]
    \arrow[from=4-12, to=3-10]
    \arrow[shorten <=5pt, Rightarrow, from=4-12, to=4-14]
    \arrow[from=4-17, to=2-14]
    \arrow[from=4-18, to=2-15]
    \arrow[from=5-3, to=5-2]
    \arrow[from=5-4, to=5-3]
    \arrow[from=5-5, to=5-4]
    \arrow[from=5-6, to=5-5]
    \arrow[from=5-10, to=4-8]
    \arrow[from=5-11, to=4-9]
    \arrow[from=5-12, to=4-10]
    \arrow[from=5-17, to=3-14]
    \arrow[from=5-18, to=3-15]
    \arrow[from=6-3, to=6-2]
    \arrow[from=6-4, to=6-3]
    \arrow[from=6-5, to=6-4]
    \arrow[from=6-6, to=6-5]
    \arrow[from=6-10, to=5-8]
    \arrow[from=6-11, to=5-9]
    \arrow[from=6-12, to=5-10]
    \arrow[from=6-17, to=4-14]
    \arrow[from=6-18, to=4-15]
\end{tikzcd}
\]
\end{remark}

\begin{definition}
    Eine spektrale Sequenz heißt Erste-Quadranten-Spektralsequenz, falls für ein $r \in \mathbb{N}$ die Einträge, also die Homologiegruppen auf der $r$ten Seite ungleich Null sind, genau dann wenn $p \geq 0$ und $q \geq 0$.
\end{definition}

\section{Filtrierungen und exakte Paare}
Angenommen wir haben einen topologischen Raum $X$ und einen guten Teilraum $A$, wobei das Gütekriterium in diesem Fall besagt, dass es eine Umgebung $U \subset X$ gibt, mit $\overline{A} \subset \mathring{U}$, so dass $A$ ein Deformationsretrakt von $U$ ist. Dies garantiert, dass die relative Homologie $H_n(X,A) \cong \tilde{H}_n(X/A)$ isomorph ist zur reduzierten Homologie des Quotienten $X/A$. Falls wir nun die Homologien von $A$ und $X/A$ kennen, so ist der natürliche Weg die lange exakte reduzierte Homologiesequenz zu benutzen, um die Homologien von $X$ zu berechnen:
\begin{align}
    \cdots \to \tilde{H}_n(A) \to \tilde{H}_n(X) \to \tilde{H}_n(X/A) \to H_{n-1}(A) \to H_{n-1}(X) \to \cdots.
\end{align}
Wir erweitern unsere Situation auf drei ineinanderliegende Teilräume $B \subset A \subset X$, wobei wir wieder annehmen, dass die Teilräume gutartig sind und wir die Homologien von $B, A/B$ und $X/A$ kennen. Dann hätten wir gerne eine Verallgemeinerung der exakten Sequenz, die auch für ein solches Paar ineinanderliegender Räume funktioniert. Allgemeiner können wir eine solche Matroschka von topologischen Räumen als Filtrierung definieren:

\begin{definition}
    Eine Filtrierung eines topologischen Raums $X$ ist eine aufsteigende Folge von topologischen Teilräumen $\mathcal{X}: \cdots X_{-2} \subset X_{-1} \subset X_{0} \subset X_{1} \subset \cdots \subset X$.
\end{definition}

Was können wir über die Homologien von $X$ sagen, indem wir unser Wissen über die Homologien der $X_i$ aus einer Filtrierung $\mathcal{X}$ und/oder den Quotienten $X_i/X_{i-1}$ verwenden? Falls wir uns auf einen Spezialfall beschränken, nämlich dass $X$ der Totalraum einer Faserung ist, so erhalten wir eine zufriedenstellende Antwort auf diese Frage.

In einer spektralen Sequenz kommt es häufig vor, dass $E^r$ und $d^r$ aus einer anderen Struktur heraus gebildet werden, einem exakten Paar.

\begin{definition}
    Ein exaktes Paar ist ein Paar bigraduierter abelscher Gruppen $A$ und $E$ zusammen mit bigraduierten Abbildungen $i: A \to A$, $j: A \to E$ und $k: E \to A$, so dass das folgende Diagramm exakt ist:
    \[\begin{tikzcd}
    	A && A \\
    	& E.
    	\arrow["i", from=1-1, to=1-3]
    	\arrow["j", from=1-3, to=2-2]
    	\arrow["k", from=2-2, to=1-1]
    \end{tikzcd}\]
    D.h.
    \begin{itemize}[nolistsep]
        \item $\ker i = \ima k$,
        \item $\ker j = \ima i$ und
        \item $\ker k = \ima j$.
    \end{itemize}
\end{definition}

Haben wir ein solches exaktes Paar gegeben, so gibt es eine natürliche Art und Weise (welche auch eine natürliche Transformation in den entsprechenden Kategorien bildet), wie man das Differential $d = jk: E \to E$ definiert, so dass $d^2 = jkjk = j(kj)k = 0$. Das Differential wird dann verwendet um ein zweites exaktes Paar zu definieren, das derivierte Paar.

\begin{lemma}
    Sei $(A,E,i,j,k)$ ein exaktes Paar, dann gibt es ein zweites exaktes Paar $(A',E',i',j',k')$, welches deriviertes Paar genannt wird, so dass
    \[\begin{tikzcd}
    	A' && A' \\
    	& E
    	\arrow["i'", from=1-1, to=1-3]
    	\arrow["j'", from=1-3, to=2-2]
    	\arrow["k'", from=2-2, to=1-1]
    \end{tikzcd}\]
    definiert ist durch
    \begin{itemize}[nolistsep]
        \item $A' = i(A)$,
        \item $E' = \ker jk / \ima jk = \ker d / \ima d$,
        \item $i' = i\vert_{i(A)}$,
        \item für alle $i(a) \in A':$ $j'(i(a)) = [j(a)] \in E'$ und
        \item für alle $[e] \in E':$ $k'([e']) = k(e)$.
    \end{itemize}
\end{lemma}

\begin{proof}
    Der Beweis gliedert sich in drei Teile. Wir prüfen die Wohldefiniertheit von $j'$ und $k'$ und die Exaktheit des Diagramms.
    \begin{enumerate}[nolistsep]
        \item Wohldefiniertheit von $j'$: Sei $i(a_1) = i(a_2)$ für $a_1,a_2 \in A$. Dann ist $a_1-a_2 \in \ker i = \ima k$ nach Exaktheit, also ist auch $j(a_1-a_2) = j(a_1)-j(a_2) \in \ima jk = \ima d$, per Definition des Differentials. Also gilt für die Äquivalenzklassen $[j(a_1)]-[j(a_2)] = 0$, da beide im Bild des Differentials liegen und deren Differenz ein Rand ist. Damit ist $[j(a_1)] = [j(a_2)]$ in $E'$ und $j'$ hängt nicht von der Wahl des Repräsentanten ab.
        \item Wohldefiniertheit von $k'$: Für jedes $e \in \ker d = \ker jk$ gilt $k(e) \in \ker j = \ima i = A'$ nach Exaktheit und der Definition $i(A') = A'$. Weiterhin ist $[e_1] = [e_2]$ für $e_1,e_2 \in E$, so gilt $[e_1 - e_2] = [0]$, also ist $e_1-e_2$ ein Rand und damit $e_1 - e_2 \in \ima d = \ima jk \subset \ima j = \ker k$. Deshalb gilt $k'([e_1] - [e_2]) = k(e_1-e_2) = 0$ und deshalb $k'([e_1]) = k'([e_2])$.
        \item Exaktheit: Sei $a' \in A'$, dann gibt es ein $a \in A$, so dass $i(a) = a'$. Wir erhalten $(j'\circ i')(a') = (j'\circ i\vert_{i(a)} \circ i)(a) = [(j \circ i\vert_{i(a)} \circ i)(a))] = [j(a'))] = [(j\circ i)(a)] = 0$, denn $ji = 0$. Somit ist $j'i' = 0$, also $\ima i' \subset \ker j'$. Für $a' \in \ker j'$ mit $a' = i(a)$, dann ist $j'(a') = j'(i(a))) = [j(a)] = 0$ und somit $j(a) \in \ima d$. Also gibt es ein $e \in E$, so dass $j(a) = d(e) = jk(e)$. Deshalb ist $a-k(e) \in \ker j = \ima i$. Also gibt es ein $b \in A$, so dass $a-k(e) = i(b)$. Weil $ik = 0$ haben wir $a' = i(a) = i(a) - (ik)(e) = i(a-k(e)) = i^2(b)$. Also ist $a' = i(i(b))$ und damit ist $a' \in \ima i'$. Also ist $\ker j' \subset \ima i'$ und deshalb $\ker j' = \ima i'$. Sei $a' = i(a) \in A'$, dann sehen wir, dass $(k'j')(a') = k'([ja]) = (kj)(a) = 0$. Also ist $\ima j' \subset \ker k'$. TODO
    \end{enumerate}
\end{proof}

Der Prozess, in dem ein deriviertes Paar von einem exakten Paar erzeugt wird, kann mehrmals iteriert werden, was uns zu einem $r$-fach derivierten Paar für ein $r \in \mathbb{N}$ führt, welches wir als $(A^r, E^r,i^r,j^r,k^r)$ bezeichnen. Sei $d^r = j^r k^r$, dann entsteht die homologische spektrale Sequenz gerade durch die paare $\{E^r, d^r\}$. Allerdings haben wir auf diese Weise noch keine Graduierung, also nur einfache abelsche Gruppen. Die Graduierung ergibt sich aus der Filtrierung $X_0 \subset X_1 \subset \cdots \subset X$, indem wir daraus ein exaktes Paar konstruieren.

\begin{definition}[Exakte Paare einer Filtrierung]
\label{exaktFiltration}
Für alle $p,q \in \Z$ sei $E^1_{p,1} = H_{p+q}(X_p,X_{p-1})$ und $A^1_{p,q} = H_{p+q}(X_p)$. Wir definieren $i^1,j^1$ und $k^1$ auf jedem $A^1$ oder $E^1$, in Vereinbarkeit mit der langen exakten Homologiesequenz:
\[\begin{tikzcd}[cramped]
	\cdots & {H_{p+q}(X_{p-1})} & {H_{p+q}(X_{p})} & {H_{p+q}(X_{p},X_{p-1})} \\
	& {} & \cdots & {H_{p+q-1}(X_{p-1})}
	\arrow["{k^1_{p,q+1}}", from=1-1, to=1-2]
	\arrow["{i^1_{p,q}}", from=1-2, to=1-3]
	\arrow["{j^1_{p,q}}", from=1-3, to=1-4]
	\arrow["{k^1_{p,q}}", from=1-4, to=2-4]
	\arrow["{i^1_{p,q-1}}"', from=2-4, to=2-3]
\end{tikzcd}\]
Die Abbildung
\begin{itemize}[nolistsep]
    \item $i^1_{p,q} = H_{p+q}(\iota): A^1 \to A^1$ wird funktoriell induziert für jedes Paar $(X_p,X_{p-1})$ durch die Inklusion $\iota: X_{p-1} \hookrightarrow X_p$.
    \item $j^1_{p,q}: A^1 \to E^1$ wird induziert durch die Quotientenabbildung auf den Kettenkomplexen und
    \item $k^1_{p,q}: E^1_{p,q} \to A^1_{p,q}$ ist durch die verbindenden Homomorphismen definiert:
    \begin{align}
        k^1_{p,q} = \partial: H_{p+q}(X_p,X_{p-1}) \to H_{p+q-1}(X_{p-1}).
    \end{align}
\end{itemize}
Da die lange exakte Homologiesequenz exakt ist, ist auch $(E^1_{p,q},A^1_{p+q},i^1_{p,q},j^1_{p,q},k^1_{p,q})$ ein exaktes Paar.
\end{definition}

Aus Def. \ref{exaktFiltration} erhalten wir ein exaktes Paar für jedes $r$, also unendlich viele exakte Paare, wobei diese Paare sowohl abzählbar, als auch überabzählbar unendlich sein können. Definieren wir $d^r = j^rk^r$, wie aus den vorhergehenden Überlegungen, dann können wir aus all diesen exakten paaren eine homologische spektrale Sequenz konstruieren.

\begin{theorem}[Homologische spektrale Sequenz einer Filtrierung]
Gegeben sei ein deriviertes exaktes Paar $(E^1_{p,q},A^1_{p+q},i^1_{p,q},j^1_{p,q},k^1_{p,q})$ aus einer homologisch exakten Sequenz einer Filtrierung, so gibt es eine homologische spektrale Sequenz $\{E^r,d^r\}$ mit $d^r = j^rk^r$.
\end{theorem}

\begin{proof}
Wir wissen bereits, dass $d^r \circ d^r = 0$ und $E^{r+1}$ ist die Homologie von $E^r$ in Bezug auf $d^r$. Also reicht es zu zeigen, dass $d^r: E^r \to E^r$ vom Grad $(-r,r-1)$ ist,. Auf der ersten Seite gilt $d^1=j^1k^1$ und damit für $(H_{p+1}(X_p,X_{p-1}), d^1)$ auch
\begin{align}
    H_{p+q}(X_p,X_{p-1}) \xrightarrow{k^1} H_{p+q-1}(X_{p-1}) \xrightarrow{j^1} H_{p+q-1}(X_{p-1},X_{p-2}).
\end{align}
Also sehen wir, dass $d^1: E^1_{p,q} = H_{p+q}(X_p,X_{p-1}) \to H_{p+q-1}(X_{p-1},X_{p-2}) = E^1_{p-1,q}$. Damit ist $d^1$ vom Grad $(-1,0)$.

Für den allgemeinen Fall gliedern wir den Beweis in drei Teile:
\begin{enumerate}[nolistsep]
    \item $A^r_{p,q} = i^{r-1}(A^1_{p,q})$.
    \item $k_{p,q}^r: E^r_{p,q} \to A^r_{p-1,q}$.
    \item $j_{p,q}^r: A^r_{p,q} \to E^r_{p-r+1,q+r-1}$.
\end{enumerate}
Nun zeigen wir diese Schritte:
\begin{enumerate}[nolistsep]
    \item Das ist ersichtlich aus der Definition von derivierten Paaren.
    \item Das ist ersichtlich, da $k^r+1_{p,q}$ durch $k^r_{p,q}$ auf $\ker d^r / \ima d^r$ induziert wird. Wir wissen bereits, dass $k^1_{p,q}: E^1_{p,q} \to A^1_{p-1,q}$, also können wir induktiv auch $k^{n}_{p,q}: E^{n-1}_{p,q} \to A^{n-1}_{p-1,q}$ definieren, für beliebiges $r$.
    \item Sei $a \in A^r_{p,q}$, dann gibt es ein $b \in A^1_{p,q}$, so dass $i^{r-1}(b) = a$, weil $A^r_{p,q} = i^{r-1}(A^1_{p,q})$ nach der ersten Forderung für derivierte Paare. Damit haben wir
    \begin{align}
    j^r(a)=j^r(i^{r-1}(b)) = [j^{r-1}(i^{r-2}(b))] = [[j^{r-2}(i^{r-3}(b))]] = \cdots = [\cdots[j^1(b)]\cdots].
    \end{align}
    Indem wir mehrere Klammern gesetzt haben, d.h. $[[\alpha]]$ für ein $\alpha \in E^{r-1}_{p,1}$, haben wir gekennzeichnet, dass wir zunächst eine Äquivalenzklasse $[\alpha] \in E^r_{p,q}$ betrachten und dann Äquivalenzklassen von Äquivalenzklassen $[[\alpha]] \in E^{r+1}_{p,q}$. Weil $[\cdots[j^1(b)]\cdots] E^{r+k}_{p,q}$ und $j^1(b) \in E^1_{p,q}$, ist die homologische Ordnung dieser beiden Äquivalenzklassen dieselbe, also nur der Grad der derivierten Paarung $r$ unterscheidet sich. Damit ist auch die homologische Ordnung von $j^r(a)$ und $j^1(b)$ gleich, denn $b \in i^{r-1}(A^r_{p,q}) = A^1_{p-r+1,q+r-1}$ und $j^1(b) \in E^r_{p-r+1,q+r-1}$. Also ist $j^r_{p,q}: A^r_{p,q} \to E^r_{p-r+1,q+r-1}$. Durch Verkettung erhalten wir $d^r: E^r_{p,q} \xrightarrow{k^r} A^r_{p-1,q} \xrightarrow{j^r} E^r_{p-r,q+r-1}$.
\end{enumerate}
Damit ist der Grad der bigraduierten Abbildung $d^r$ gleich $(-r,r-1)$.
\end{proof}

\section{Presentation der Serre spektralen Sequenz}
Die Serre spektrale Sequenz ist konstruiert worden um die Homologiegruppen von unterschiedlichen Bestandteilen einer sogenannten Faserung zusammenzubringen. Bei Einer Faserung $Z \to X \to Y$ sollte man an eine Abbildung $pi: X \to Y$ denken, sodass alle Urbilder $\pi^{-1}(y)$ für $y \in Y$ homotopieäquivalent zu $Z$ sind. Bevor wir die Presentation der Serre spektralen Sequenz einführen können, benötigen wir also die Definition einer Faserung, welche die Homotopiehochhebungseigenschaft erfüllt. Diese erlaubt uns durch Anwendung der Faserung aus einfachen, bekannten Räumen, neue komplexere Räume zu konstruieren. Die Faserung ist eine Verallgemeinerung des Vektorbündels, welche auch ohne lineare oder affine Struktur auskommt. Die Homotopie-Hochhebungseingeschaft wird uns anschließend auch kompatible Eigenschaften auf Homologieebene induzieren.

\begin{definition}[Faserung]
Eine Abbildung $\pi: X \to B$, welche folgende Homotopiehochhebungseigenschaften für beliebige Räume $X$ erfüllt, heißt Faserung:
\begin{itemize}[noitemsep]
    \item für jede Homotopie $h: Z \times [0,1] \to B$ und
    \item für jeden Lift $\overline{h}_0: Z \to X$ mit $h_0 = h\vert_{X\times \{0\}}$ und $h_0 = \pi \circ \overline{h}_0$,
\end{itemize}
existiert eine Homotopie $\overline{h}: Z \times [0,1] \to X$, die $h$ liftet. Daraus resultiert folgendes kommutierende Diagramm:
\[\begin{tikzcd}
    {Z \times \{0\}} & X \\
    {Z\times[0,1]} & B.
    \arrow["{\overline{h}_0}", from=1-1, to=1-2]
    \arrow["\iota"', hook, from=1-1, to=2-1]
    \arrow["{\overline{h}}", dashed, from=2-1, to=1-2]
    \arrow["h", from=2-1, to=2-2]
    \arrow["\pi"', from=2-2, to=1-2]
\end{tikzcd}\]
\end{definition}
Im folgenden betrachten wir ein paar Beispiele von Faserungen, um etwas Intuition zu gewinnen.
\begin{example} Beispiele für Faserungen:
\begin{enumerate}[nolistsep]
    \item Seien $X,Y$ topologische Räume, dann ist die Projektion $\pi: X \times Y \to Y$ eine Faserung, wobei jede Faser homeomorph ist zu $X$. Sei $h: Z \times [0,1] \to Y$ eine Homotopie mit $\overline{h}_0: Z \times \{0\} \to X \times Y$, so dass $(\pi \circ \overline{h}_0)(z) = h_0(z) = h(z,0)$. Sei $\overline{h}: Z \times [0,1] \to X \times Y$ definiert durch $\overline{h}(z,t) = ((p_X \circ \overline{h}_0)(z,t), h(z,t))$, wobei $p_X: X \times Y \to X$ die Projektion auf die erste Komponente bezeichnet. Wir haben also $\overline{h}$ so gebaut, dass es mit $\overline{h}_0$ in der $X$-Komponente und mit $h$ in der $Y$-Komponente übereinstimmt.
    \item Die kanonische Surjektion $\pi: S^n \to \mathbb{R}P^n, x \mapsto [x]$ ist eine Faserung mit Fasern homeomorph zu $S^0$.
    \item Die kanonische Surjektion $\pi: S^{2n+1} \to \mathbb{C}P^n, x \mapsto [x]$ ist eine Faserung mit Fasern homeomorph zu $S^1$.
\end{enumerate}
\end{example}

Als gegebene Daten verwenden wir eine Faserung $\pi: X \to B$, wobei $B$ ein wegzusammenhängender CW-Komplex ist und $X$ ein topologischer Raum. Man beachte, dass wir die Homotopiehochhebungseigenschaften nur für CW-Komplexe fordern wollen, eine solche Faserung nennt man auch Serre-Faserung. Der Raum $B$ wird Basisraum genannt und $X$ heißt Totalraum. Wir können eine Filtrierung von $X$ erzeugen, indem wir das $p$-Skelett $B^p$ von $B$ betrachten und $X_p \coloneq \pi^{-1}(B^p)$ definieren. Aus dieser Filtrierung erhalten wir ein exaktes Paar und von diesem exakten Paar wiederum eine homologische Sequenz. In dieser Konstellation heißt die resultierende spektrale Sequenz auch Serre spektrale Sequenz.

Wir benötigen zunächst für einen Weg $\gamma: [0,1] \to B$ in den Basisraum $B$ eine Abbildung $L_\gamma: F_a \to F_b$ für die Fasern $F_a = \pi^{-1}(a)$ und $F_b = \pi^{-1}(b)$. Diese Abbildung zwischen den Fasern soll stetig sein und die Existenz einer solchen Abbildung wird garantiert, durch die Homotopieäquivalenz der Fasern über jede Wegzusammenhangskomponenten zueinander.

\begin{proposition}
Falls $\pi: X \to B$ eine Faserung ist, dann sind die Fasern $F_b = \pi^{-1}(b)$ über jede Wegzusammenhangskomponente von $B$ zueinander homotopieäquivalent.
\end{proposition}

\begin{proof}
Sei $\gamma: [0,1] \to B$ ein Weg und $F_{\gamma(s)} \coloneq \pi^{-1}(\gamma(s))$ eine Faser. Wir definieren eine Homotopie $h: F_{\gamma(0)} \times [0,1] \to B$ durch $h(x,t) = \gamma(t)$. Sei $\overline{h}_0: F_{\gamma(0)} \times \{0\} \hookrightarrow X$ die Inklusion. Dann gilt für $x \in F_{\gamma(0)}, (\pi \circ \overline{h}_0)(x) = \gamma(0) = h_0(x)$. Da $\pi: X \to B$ eine Faserung ist, existiert eine Abbildung $\overline{h}: F_{\gamma(0)} \times [0,1] \to X$, sodass das folgende Diagramm kommutiert:
\[\begin{tikzcd}
    {F_{\gamma(0)} \times \{0\}} & X \\
    {F_{\gamma(0)}\times[0,1]} & B.
    \arrow["{\overline{h}_0}", from=1-1, to=1-2]
    \arrow["\iota"', hook, from=1-1, to=2-1]
    \arrow["{\overline{h}}", dashed, from=2-1, to=1-2]
    \arrow["h", from=2-1, to=2-2]
    \arrow["\pi"', from=2-2, to=1-2]
\end{tikzcd}\]
Insbesondere gilt $h = \pi \overline{h}, \overline{h}(x,t) \in F_{\gamma(t)}$. Falls also $t=1$ ist, erhalten wir die Abbildung $L_\gamma: F_{\gamma(0)} \to F_{\gamma(1)}$ definiert durch $L_\gamma(x) = \overline{h}(x,1)$.

Wir zeigen als nächstes, dass die Abbildung $\gamma \mapsto L_\gamma$ sich gutartig verhält, in Bezug auf Homotopien \cite[Prop. 4.61]{hatcher2001}:
\begin{itemize}[noitemsep]
    \item Falls $\gamma_0$ und $\gamma_1$ Wege von $a$ nach $b$ sind und $\gamma \simeq_{\{0,1\}} \gamma'$ (d.h. es gibt eine Homotopie $h: \gamma \Rightarrow \gamma'$, welche konstant auf den Basispunkten $0$ und $1$ ist), dann gilt $L_\gamma \simeq L_{\gamma'}$.
    \item Falls $\gamma_0$ und $\gamma_1$ Wege sind, so dass $\gamma_1(0) = \gamma_0(1)$, dann ist $L_{\gamma_0 \star y_1} \simeq L_{\gamma_1} \circ L_{\gamma_0}$.
\end{itemize}
Wir werden nun die Proposition aus diesen beiden Eigenschaften ableiten. Seien $F_a$ und $F_b$ Fasern und sei $\gamma$ ein Weg von $a$ nach $b$. Wir betrachten $L_\gamma: F_a \to F_b$ und $L_{\overline{\gamma}}: F_b \to F_a$.

Sei $\alpha \equiv a$ die Konstante Schleife und $\overline{h}$ die Abbildung, welche wir verwenden um $L_\alpha(x)$ zu definieren. Weil $h(x,t) = \alpha(x) = a$ ist für alle $x$, wissen wir auch dass $\overline{h}(x,t) \in F_a$ für alle $x$ und $t$. Also ist $\overline{h}$ eine Homotopie, so dass $\overline{h}(x,0) = \operatorname{Id}_{F_a}$ und $\overline{h}(x,1) = L_\alpha$. Also ist auch $\operatorname{F}_a \simeq L_\alpha$ und ebenso $\operatorname{Id}_{F_b} \simeq L_\beta$.

Mit den Eigenschaften aus \cite[Prop. 4.61]{hatcher2001} erhalten wir
\begin{align}
L_{\overline{\gamma}} \circ L_\gamma &\simeq L_{\gamma \star \overline{\gamma}} \simeq L_\alpha \simeq \operatorname{Id}_{F_a}, \\
L_{\gamma} \circ L_{\overline{\gamma}} &\simeq L_{\overline{\gamma} \star \gamma} \simeq L_\beta \simeq \operatorname{Id}_{F_b}.
\end{align}
Also ist $F_a \simeq F_b$.
\end{proof}

Für jeden Weg $\gamma: [0,1] \to B$ können wir also eine Abbildung zwischen den Fasern der Anfangs- und Endpunkte definieren $L_\gamma: F_{\gamma(0)} \to F_{\gamma(1)}$, indem wir die Homotopiehochhebungseigenschaft in Bezug auf $F_{\gamma(0)}$ verwenden. 

\section{Homologie von $\mathbb{C}P^\infty$}


Etwas seltsam verhält es sich mit der Serre spektralen Sequenz für die Homologie von $\mathbb{C}P^\infty$, denn wir ignorieren im Grunde die erste Seite $E^1$ und beginnen unsere Berechnungen mit $E^2$ und setzen diese auf alle weiteren Seiten fort. Der Grund dafür ist, das wir eine sehr einfache aber sehr nützliche Fomel haben, für die Gruppen auf der zweiten Seite. Im Fall unserer Faserung $S^1 \to S^\infty \to \mathbb{C}P^\infty$ können wir die ganze zweite Seite $E^2$ sofort ausschreiben.

Nach Teil c) von Satz 
\nocite{*}
\printbibliography
\end{document}
